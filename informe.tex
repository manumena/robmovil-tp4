% ALGUNOS PAQUETES REQUERIDOS (EN UBUNTU): %
% ========================================
% %
% texlive-latex-base %
% texlive-latex-recommended %
% texlive-fonts-recommended %
% texlive-latex-extra %
% texlive-science %
% texlive-lang-spanish (en ubuntu 13.10) %
% ******************************************************** %

\documentclass[a4paper]{article}
\usepackage[spanish, es-nodecimaldot]{babel}
\usepackage[utf8]{inputenc}
\usepackage{fancyhdr}
\usepackage[pdftex]{graphicx}
\usepackage{sidecap}
\usepackage{caption}
\usepackage{subcaption}
\usepackage{booktabs}
\usepackage{makeidx}
\usepackage{float}
\usepackage{amsmath, amsthm, amssymb}
\usepackage{amsfonts}
\usepackage{sectsty}
\usepackage{wrapfig}
\usepackage{listings}
\usepackage{pgfplots}
\usepackage{pgfplotstable}
\usepackage{enumitem}
\usepackage[hidelinks]{hyperref}
\usepackage{listings}
\usepackage{listingsutf8}

\linespread{factor}

\definecolor{mygreen}{rgb}{0,0.6,0}
\definecolor{mygray}{rgb}{0.5,0.5,0.5}
\pgfplotsset{compat=1.8}
\setlist[enumerate]{label*=\arabic*.}
\lstset{
	inputencoding=utf8/latin1,
	language=C++,
	basicstyle=\ttfamily,
	keywordstyle=\bfseries\color{blue},
	stringstyle=\color{red}\ttfamily,
	commentstyle=\color{mygreen}\ttfamily,
	morecomment=[l][\color{magenta}]{\#},
	numbers=left,
	numberstyle=\color{mygray}
}

\usepackage{fancyhdr}
\pagestyle{fancy}
\fancyhf{}
\fancyhead[LO]{Introducción a la robótica móvil}
\fancyhead[RO]{Trabajo Práctico N\textsuperscript{o} 3}
%\fancyfoot[LO]{\small{Shai Bianchi, Martín Jedwabny, Manuel Mena, Iván Pondal}}
\fancyfoot[RO]{\thepage}
\renewcommand{\headrulewidth}{0.5pt}
\renewcommand{\footrulewidth}{0.5pt}
\setlength{\textwidth}{16cm}
\setlength{\hoffset}{-1.1cm}
\setlength{\headsep}{0.5cm}
\setlength{\textheight}{25cm}
\setlength{\voffset}{-1.75cm}
\setlength{\headwidth}{\textwidth}
\setlength{\headheight}{13.1pt}
\renewcommand{\baselinestretch}{1.1} % line spacing

\usepackage{caratula}

\allowdisplaybreaks
\newcommand{\ord}{\ensuremath{\operatorname{O}}}
\newcommand{\nat}{\ensuremath{\mathbb{N}}}
\newcommand{\real}{\ensuremath{\mathbb{R}}}
\newcommand{\acr}[1]{\lowercase{\textsc{#1}}}
\newcommand{\comp}{\ensuremath{^{\operatorname{C}}}}
\newcommand{\argmax}{\operatornamewithlimits{arg\,m\acute{a}x}}

\newcommand{\subheading}[1]{\vspace{1em} \noindent\textbf{#1} \nopagebreak
\smallskip \nopagebreak}

% Lemas, definiciones, etc.
\theoremstyle{plain}
  \newtheorem{prop}{Proposición}
  \newtheorem{lema}{Lema}
\theoremstyle{remark}
  \newtheorem{obs}{Observación}
\theoremstyle{definition}
  \newtheorem{defi}{Definición}

% Pseudocódigo
\usepackage[onelanguage, spanish]{algorithm2e}
    % \NoCaptionOfAlgo
    \LinesNumbered\RestyleAlgo{ruled}\IncMargin{1em}\DontPrintSemicolon
    \SetArgSty{}\SetCommentSty{textsf}\SetFuncSty{textsf}
    \SetKwInput{Input}{Entrada}
    \SetKwInput{Output}{Salida}
    \SetKwProg{For}{para}{ hacer}{fin}
    \SetKwProg{Fn}{función}{:}{fin}

\begin{document}
\materia{Introducción a la Robótica Móvil}
\submateria{Segundo cuatrimestre de 2016}
\titulo{Trabajo Práctico N\textsuperscript{o} 4}
\subtitulo{Planificación de caminos utilizando RRT}
\integrante{Luis García Gómez}{675/13}{garcia\_luis\_94@hotmail.com}
\integrante{Manuel Mena}{313/14}{manuelmena1993@gmail.com}

\maketitle
% no footer on the first page
\thispagestyle{empty}
\newpage

\section{Ejercicio 1}

\subsection{Explicar cómo definieron el ``área cercana al goal''. ¿Tomaron en cuenta el
ángulo $\theta$? ¿Cómo?}

Definimos el área cercana al goal como un cuadrado de lado 0.5 m con centro en
el goal. Sí, tomamos en cuenta el ángulo del goal y le sumamos un valor
aleatorio entre $\frac{-\pi}{4}$ y $\frac{\pi}{4}$.

\subsection{Explicar que definición de distancia utilizaron. ¿Cómo integran el ángulo $\theta$?}

Utilizamos una función de distancia propuesta en clase: $d(q_1, q_2) = k_{pos} * d_{(x, y)}(q_1, q_2) + k_{or} * d_{\theta}(q_1, q_2)$, donde: \\

\begin{itemize}
  \item{$q_1, q_2$ son dos configuraciones: cada cual describe una posición en $(x, y)$ y una orientación $\theta$.}
  \item{$d_{(x, y)}$ es una función que dadas dos configuraciones retorna la distancia euclídea entre los puntos en el plano que describen.}
  \item{$d_{\theta}$ es una función que dadas dos configuraciones retorna la diferencia normalizada entre las orientaciones que describen.}
  \item{$k_{pos}$ es una constante real que determina el peso de $d_{(x, y)}$ relativo a $d_{\theta}$.}
  \item{$k_{or}$ es una constante real que determina el peso de $d_{\theta}$ relativo a $d_{(x, y)}$.}
\end{itemize}

\subsection{Explicar como establecieron ”discretizaron” el espacio de posibilidades a partir
de la ”configuración más cercana”.}

Cómo el enunciado del trabajo proponía, para ”discretizar” el espacio de posibilidades se aplicaron algunas transformaciones simples a la velocidad lineal en $x$ ($v_x$) y la velocidad angular en $z$ ($w_z$) del robot. De esta manera establecimos tres configuraciones posibles candidatas a ser el resultado del método $steer()$. \\

Sea $(x_r, y_r, \theta_r)$ la configuración más cercana encontrada, las transformaciones que aplicamos fueron: \\

Adelante-Izquierda: \\

\begin{itemize}
  \item{$x = x_r + v_x * cos(\theta_r + w_z)$}
  \item{$y = y_r + v_x * sin(\theta_r + w_z)$}
  \item{$\theta = \theta_r + w_z$}
\end{itemize}

Adelante-Centro: \\

\begin{itemize}
  \item{$x = x_r + v_x * cos(\theta_r)$}
  \item{$y = y_r + v_x * sin(\theta_r)$}
  \item{$\theta = \theta_r$}
\end{itemize}

Adelante-Derecha: \\

\begin{itemize}
  \item{$x = x_r + v_x * cos(\theta_r - w_z)$}
  \item{$y = y_r + v_x * sin(\theta_r - w_z)$}
  \item{$\theta = \theta_r - w_z$}
\end{itemize}

\subsection{Explicar como resolvieron esta comprobación.}

Para definir el método $isFree()$ nos basamos en lo siguiente: ajustamos el perímetro del robot a un rectángulo y verificamos que las esquinas de dicho rectángulo no estén en alguna posición que esté ocupada. Para esto último utilizamos el método propuesto $isPositionOccupy()$.

\end{document}